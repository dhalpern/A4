\documentclass{article} % For LaTeX2e
\usepackage{nips14submit_e,times}
\usepackage{amsmath}
\usepackage{fancyvrb}
\usepackage{hyperref}
\usepackage{url}
%\documentstyle[nips14submit_09,times,art10]{article} % For LaTeX 2.09


\title{A4 homework submission \\ David Halpern \\ Deep Learning 2015, Spring}


\author{
David Halpern\\
Department of Psychology\\
New York University\\
\texttt{david.halpern@nyu.edu} \\
}

\newcommand{\fix}{\marginpar{FIX}}
\newcommand{\new}{\marginpar{NEW}}

\nipsfinalcopy % camera-ready version

\begin{document}


\maketitle


\section{Questions}

\subsection{Q2}

i = i\\
$\text{prev\_c} = c_{t-1}^l$\\
$\text{pre\_h} = h_{t-1}^l$

\subsection{Q3}
Returns an rolled? version of the core LSTM module with parameters initialized uniformly within an interval.

\subsection{Q4}
model.s is the trained of the sequence. model.ds is the gradients of the model for the backward pass. model.start$\_$s is the final hidden states of the current minibatch and gets reset when the sequence ends

\subsection{Q5}
Clipped at params.max$\_$grad$\_$norm (shrink factor?)

\subsection{Q6}
Batch SGD

\subsection{Q7}
Since the prediction depended on how the output was being used (multinomial vs.

\section{Model}
\subsection{Architecture}
The basic model used was very similar to the one used by Zaremba (2014). 

\section{Experiments}
\subsection{1} 
Network parameters:	
\begin{Verbatim}
{
  max_grad_norm : 7
  seq_length : 50
  batch_size : 50
  lr : 1
  max_max_epoch : 13
  rnn_size : 400
  init_weight : 0.1
  decay : 2
  dropout : 0.2
  layers : 2
  vocab_size : 50
  max_epoch : 4
}
\end{Verbatim}

\subsubsection*{References}

\small{
% [1] Dosovitskiy, A., Springenberg, J. T., Riedmiller, M., \& Brox, T. (2014). Discriminative unsupervised feature learning with convolutional neural networks. {\it Advances in Neural Information Processing Systems}, pp. 766-774.

[1] Pennington, Jeffrey, Socher, Richard, \& Manning, Christopher D. (2014). Glove: Global vectors for word representation. Proceedings of the Empirical Methods in Natural Language Processing.

[2] Xiang Zhang and (2015). Text Understanding from Scratch. CoRR, abs/1502.01710, .
}


\end{document}
